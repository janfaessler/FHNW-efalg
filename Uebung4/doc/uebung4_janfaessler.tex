\documentclass[10pt]{scrartcl}

%Math
\usepackage{amsmath}
\usepackage{amsfonts}
\usepackage{amssymb}
\usepackage{amsthm}
\usepackage{ulem}
\usepackage{stmaryrd} %f\UTF{00FC}r Blitz!

%PageStyle
\usepackage[ngerman]{babel} % deutsche Silbentrennung
\usepackage[utf8]{inputenc} 
\usepackage{fancyhdr, graphicx}
\usepackage[scaled=0.92]{helvet}
\usepackage{enumitem}
\usepackage{parskip}
\usepackage[a4paper,top=2cm]{geometry}
\setlength{\textwidth}{17cm}
\setlength{\oddsidemargin}{-0.5cm}
\usepackage[scaled=0.92]{helvet}
\usepackage{lastpage} % for getting last page number
\renewcommand{\familydefault}{\sfdefault}


% Shortcommands
\newcommand{\Bold}[1]{\textbf{#1}} %Boldface
\newcommand{\Kursiv}[1]{\textit{#1}} %Italic
\newcommand{\T}[1]{\text{#1}} %Textmode
\newcommand{\Nicht}[1]{\T{\sout{$ #1 $}}} %Streicht Shit durch

%Arrows
\newcommand{\lra}{\leftrightarrow} 
\newcommand{\ra}{\rightarrow}
\newcommand{\la}{\leftarrow}
\newcommand{\lral}{\longleftrightarrow}
\newcommand{\ral}{\longrightarrow}
\newcommand{\lal}{\longleftarrow}
\newcommand{\Lra}{\Leftrightarrow}
\newcommand{\Ra}{\Rightarrow}
\newcommand{\La}{\Leftarrow}
\newcommand{\Lral}{\Longleftrightarrow}
\newcommand{\Ral}{\Longrightarrow}
\newcommand{\Lal}{\Longleftarrow}

% Code listenings
\usepackage{color}
\usepackage{xcolor}
\usepackage{listings}
\usepackage{caption}
\DeclareCaptionFont{white}{\color{white}}
\DeclareCaptionFormat{listing}{\colorbox{gray}{\parbox{\textwidth}{#1#2#3}}}
\captionsetup[lstlisting]{format=listing,labelfont=white,textfont=white}
\lstdefinestyle{JavaStyle}{
 language=Java,
 basicstyle=\footnotesize\ttfamily, % Standardschrift
 numbers=left,               % Ort der Zeilennummern
 numberstyle=\tiny,          % Stil der Zeilennummern
 stepnumber=5,              % Abstand zwischen den Zeilennummern
 numbersep=5pt,              % Abstand der Nummern zum Text
 tabsize=2,                  % Groesse von Tabs
 extendedchars=true,         %
 breaklines=true,            % Zeilen werden Umgebrochen
 frame=b,         
 %commentstyle=\itshape\color{LightLime}, Was isch das? O_o
 %keywordstyle=\bfseries\color{DarkPurple}, und das O_o
 basicstyle=\footnotesize\ttfamily,
 stringstyle=\color[RGB]{42,0,255}\ttfamily, % Farbe der String
 keywordstyle=\color[RGB]{127,0,85}\ttfamily, % Farbe der Keywords
 commentstyle=\color[RGB]{63,127,95}\ttfamily, % Farbe des Kommentars
 showspaces=false,           % Leerzeichen anzeigen ?
 showtabs=false,             % Tabs anzeigen ?
 xleftmargin=17pt,
 framexleftmargin=17pt,
 framexrightmargin=5pt,
 framexbottommargin=4pt,
 showstringspaces=false      % Leerzeichen in Strings anzeigen ?        
}

 
\fancypagestyle{firststyle}{ %Style of the first page
\fancyhf{}
\fancyheadoffset[L]{0.6cm}
\lhead{
\includegraphics[scale=0.8]{./fhnw_ht_e_10mm.jpg}}
\renewcommand{\headrulewidth}{0pt}
\lfoot{Institute of computer science,\linebreak www.fhnw.ch }
}

\fancypagestyle{documentstyle}{ %Style of the rest of the document
\fancyhf{}
\fancyheadoffset[L]{0.6cm}
\lhead{
\includegraphics[scale=0.8]{./fhnw_ht_e_10mm.jpg}}
\renewcommand{\headrulewidth}{0pt}
\lfoot{\thepage\ / \pageref{LastPage} }
}

\pagestyle{firststyle} %different look of first page
 
\title{ %Titel
Uebung 4 \\
\vspace{0.2cm} 
\Large Jan Fässler }

 \begin{document}
 \maketitle
 \pagestyle{documentstyle}
\section{Einleitung}
Dieses Dokument erklärt meinen Lösungsansatz der Uebung 4 im Modul Effiziente Algorithmen. Meine Lösung behandelt die  \textbf{Variante A} der Übung.
\section{Aufgabestellung}
Entwickeln und implementieren Sie einen Algorithmus, welcher eine Anzahl Punkte in der xy–Ebene (durch Mouse–Klicks in einem einfach GUI definiert) mit einem möglichst kleinen Rechteck umschliesst.
\section{Algorithmus}
Der Algorithmus funktioniert so, dass er die konvexe Hülle der Punkte betrachtet und diese mit Linien verbindet. Für diese Linien wird der Winkel berechnet, in dem sie zur X-Achse stehen. Die Punkte der konvexen Hülle werden jeweils um den negativen Wert der Winkel gedreht. Dadurch kann das Rechteck einfach durch das herausfinden der min/max Werte der X- und Y-Achse bestummen und dessen Fläche berechnet werden. \\
Von diesen Rechtecken wird dan dasjenige genommen, welches die kleinste Fläche hat. Durch rotieren der Eckpunkte um den Winkel der Linie aus der konvexen Hülle kann das Rechteck einfach gefunden werden.
\section{Implementierung}
Die Implementierung des Algorithmus ist in der Rectangle Klasse gemacht. Diese Klasse kann entweder mit 4 Double Werte (min/max für X- und Y-Achse) oder mit einer Liste von Punkten erzeugt werden. Wird der Konstruktor mit einer Liste von Punkten gefüttert und er definiert das Rechteck mit der Hilfe des Algorithmus. \\
Aus Gründen der Effizienz, darf die Liste nur die Punkte auf der konvexen Hülle beinhalten. In der Application Klasse wird beim Einfügen eines neuen Punktes nur der neue Punkt und die Punkte der alten konvexen Hülle genommen für die Berechnung der neuen Hülle. \\
Die konvexe Hülle wird mit der JarvisMarch Implementation von Manfred Vogel berechnet.
\section{Fazit}
Der Algorithmus läuft mit einer Laufzeit von $O(n^2)$. Wobei n die Anzahl der Punkte auf der konvexen Hülle repräsentiert.


 \end{document}